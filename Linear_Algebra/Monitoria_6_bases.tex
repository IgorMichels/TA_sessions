\documentclass[12pt,letterpaper]{article}

\usepackage[brazilian]{babel}
\usepackage[utf8]{inputenc}
\usepackage[T1]{fontenc}

\usepackage{fullpage}
\usepackage[top=2cm, bottom=4.5cm, left=2.5cm, right=2.5cm]{geometry}
\usepackage{amsmath,amsthm,amsfonts,amssymb,amscd}
\usepackage{lastpage}
\usepackage{enumerate}
\usepackage{fancyhdr}
\usepackage{mathrsfs}
\usepackage{xcolor}
\usepackage{graphicx}
\usepackage{listings}
\usepackage{hyperref}

\hypersetup{%
  colorlinks=true,
  linkcolor=blue,
  linkbordercolor={0 0 1}
}
 
\renewcommand\lstlistingname{Algorithm}
\renewcommand\lstlistlistingname{Algorithms}
\def\lstlistingautorefname{Alg.}

\lstdefinestyle{Python}{
    language        = Python,
    frame           = lines, 
    basicstyle      = \footnotesize,
    keywordstyle    = \color{blue},
    stringstyle     = \color{green},
    commentstyle    = \color{red}\ttfamily
}

\setlength{\parindent}{0.0in}
\setlength{\parskip}{0.05in}

% Edit these as appropriate
\newcommand\course{CSE 3500}
\newcommand\hwnumber{1}                  % <-- homework number
\newcommand\NetIDa{netid19823}           % <-- NetID of person #1
\newcommand\NetIDb{netid12038}           % <-- NetID of person #2 (Comment this line out for problem sets)

\pagestyle{fancyplain}
\headheight 35pt              % <-- Comment this line out for problem sets (make sure you are person #1)
\chead{\textbf{\Large Monitoria 6 - Bases}}
\rhead{\course \\ \today}
\lfoot{}
\cfoot{}
\rfoot{\small\thepage}
\headsep 1.5em

\begin{document}

\section*{Definições e Teoremas}

\textbf{Linearmente Independente: } Um cojunto $X \subset E$ é dito linearmente independente, quando nenhum vetor do conjunto é combinação linear dos outros vetores. O conjunto unitário é dito LI. Para isso, existe o teorema de que: $\alpha_1v_1 + ... + \alpha_nv_n = 0 \to \alpha_1 = ... = \alpha_n = 0$, se e só se, X é LI. A partir disso, conclue-se que a representação de um vetor como combinação de outros vetores é sempre única (se os vetores formarem um conjunto LI). Se um conjunto não é LI, ele é dito linearmente dependente.

\textbf{Teorema 1: } Seja $X = \{x_1,x_2,...,x_m\}$. Se, $\forall k \leq m, v_k$ não é combinação linear de seus antecessores, então X é LI. 

\textbf{Observação: } Considere $X = \{(1,2),(3,4),(2,4)\} \subset \mathbb{R}^2$, Note que $X$ é LD, porém $(3,4)$ não é combinação linear dos outros vetores (verifique!). Por que isso não é contraditório?

\textbf{Base: } É um conjunto linearmente independente que gera E. Os coeficientes são chamados de coordenadas do vetor nessa base. Como veremos a seguir, toda base de um espaço vetorial apresenta o mesmo número de elementos. Este número é chamado de \textit{dimensão}. 

\textbf{Lema 2.1: } Todo sistema homogêneo cujo número de incógnitas é maior que o número de equações admite solução não trivial (a prova é por indução em $m$, o número de equações. 

\textbf{Teorema 2.2: } Se um conjunto de n vetores gera o espaço E, então qualquer conjunto com mais de n elementos é LD.

\textbf{Corolário 2.3: } Assim, se os vetores $v_1,...,v_n$ geram o espaço vetorial $E$ e os vetores $u_1,...,u_m$ são LI, $m\leq n$. Daqui tiramos que se $E$ admite uma base $\beta = \{u_1,...,u_n\}$, qualquer outra base também possui n elementos.

\textbf{Teorema 3: } Considere um espaço vetorial de dimensão finita:
\begin{itemize}
    \item Considere o conjunto de todos os geradores de E. Ele contém uma base.
    \item Todo conjunto LI está contido numa base.
    \item Todo subespaço vetorial tem dimensão finita. 
    \item Se a dimensão de um subespaço é $n$, então o subespaço é o próprio espaço. 
\end{itemize}

\section*{Exercícios: }
\begin{enumerate}
    \item Prove que os seguintes polinômios são linearmente independentes: $p(x) = x^3 - 5x^2 + 1, q(x) = 2x^4 + 5x - 6, r(x) = x^2 - 5x + 2$. 
    \textit{Dica: Considere a base $X = \{1, x, x^2, x^3, x^4\}$} 
    \item Seja $X$ um conjunto de polinômios.  Se dois polinômios quaisquer de $X$ têm graus diferentes, $X$ é LI.
    \item Dado $X \subset E $, seja $Y$ o conjunto obtido de $X$ substituindo um dos seus elementos $v$ por $v + \alpha u$, onde $u \in X$ e $\alpha \in \mathbb{R} $. Prove que X e Y geram o mesmo subespaço vetorial de $E$. Conclua, então que $\{v_1,...,v_k\} \subset E$ e $\{v_1, v_2 - v_1, ..., v_k - v_1\} \subset E $ geram o mesmo subespaço vetorial de $E$.
    \item Mostre que os vetores $u = (1,1)$ e $v = (-1,1)$ formam uma base de $\mathbb{R}^2$. 
    \item Considere a afirmação: "A união de dois conjuntos subconjuntos LI do espaço vetorial E é ainda um conjunto LI". Assinale verdadeiro e falso. \\
    (  ) Sempre.\\
    (  ) Nunca. \\
    (  ) Quando um deles é disjunto do outro. \\
    (  ) Quanto um deles é parte do outro. \\
    (  ) Quando um deles é disjunto do subespaço gerado pelo outro. \\ 
    (  ) Quando o número de elementos de um deles mais o número de elementos do outro é igual à dimensão de E.
    \item Encontre uma base para o espaço vetorial $W = \{\begin{pmatrix}a \\ b \\ -b \\ a\end{pmatrix}, \forall a,b \in \mathbb{R}^2\}$. 
    \item Se $f$ e $g$ estão no espaço vetorial de todas as funções com derivadas contínuas, então o determinante de $\begin{pmatrix} f(x) ~~ g(x) \\ f'(x) ~~ g'(x) \end{pmatrix}$ é conhecido como \textbf{Wronskiano} de $f$ e $g$. Prove que $f$ e $g$ são linearmente independentes, se seu Wronskiano não for identicamente nulo. Esse estudo é estremamente importante no estudo de soluções de sistemas de equações diferenciáveis, pois identifica se duas soluções são linearmente independentes.   
\end{enumerate}

\section*{Aplicação: Quadrados Mágicos}

Observe a imagem da Melancolia I, de Albrecht Durer de 1514:
\href{https://artsandculture.google.com/asset/melencolia-i-albrecht-d%C3%BCrer/aAGgEK-AKbn5eQ?hl=zh-tw&ms=%7B%22x%22%3A0.7518538853310617%2C%22y%22%3A0.2894316595268001%2C%22z%22%3A10%2C%22size%22%3A%7B%22width%22%3A0.4646606018194542%2C%22height%22%3A0.355444997236042%7D%7D}{Link da obra}
Observa-se o quadrado mágico: 
$\begin{pmatrix}16~~3~~2~~13\\5~~10~~11~~8\\9~~6~~7~~12\\4~~15~~14~~1\end{pmatrix}$. Primeira coisa interessante é ver $15$ e $14$ lado a lado. A soma de cada coluna, linha e diagoral é $34$. Podemos definir uma matriz $n\times n$ sendo quadrado mágico quando a soma de cada linha, coluna e diagonal é igual. Essa soma se cham peso. Considere $Mag_n$ o conjunto de todos os quadrados mágicos de ordem $n$. Prove que $Mag_3$ é um subespaço de $M_{33}$. 

\end{document}

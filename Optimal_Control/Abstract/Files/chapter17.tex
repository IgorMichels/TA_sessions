Vamos considerar problemas especiais em que o problema é linear 
no controle $u(t)$. 

\subsection{Controle Bang-Bang}

Considere o problema de controle ótimo.

$$max_u \int_{t_0}^{t_1} f_1(t,x) + u(t)f_2(t,x) dt$$
$$s.a.~~x'(t) = g_1(t,x) + u(t)g_2(t,x), x(0) = x_0$$
$$ a \leq u(t) \leq b$$

Assim $H(t,x,u,\la) = f_1(t,x) + \la g_1(t,x) + u(t)(f_2(t,x) + \la g_2(t,x))$, 
linear em $u(t)$. Observe a derivada parcial em relação a $u$ não
carrega informação sobre $u(t)$. Assim definimos $\psi(t) := f_2(t,x(t)) + \la(t)g_2(t,x(t))$,
muitas vezes chamada de função de troca. Se $\psi = 0$ não pode ser obtido 
em um intervalo de tempo, mas ocorre apenas em pontos finitos, o controle
é dito "Bing Bang", porque só varia entre os valores mínimo e máximo de $u(t)$. 
Os valores de $u(t)$ nesses pontos não são de interesse, portanto. 

Em contrapartida, se $\psi(t) \equiv 0$ em um intervalo de tempo, dizemos que $u^*$ é 
singular nesse intervalo. Esse caso será explorado na próxima sessão. 

Para resolver esse tipo de problema, primeiro precisamos verificar se de fato 
o problema é Bang-Bang. Numericamente, o problema é apenas em verificar a positividade
ou negatividade da função de troca. 

\subsection{Controles Singulares}

O livro explora um exemplo inicial:

$$max_u \int_0^2 (x(t) - t^2)^2 dt $$
$$s.a. ~~ x'(t) = u(t), x(0) = 1$$
$$0 \leq u(t) \leq 4$$

Calculamos o Hamiltoniano e encontramos $u^*(t)$ em função da adjunta. 
Para sair dessa hipótese, precisamos fazer uma análise mais minunciosa. 
Ela começa em provar que $x(t) \geq t^2 \rightarrow \lambda'(t) \leq 0 \wedge \la(t) \geq 0$. 
Então, basta encontrar os valores de $t$ em que essa função é positiva 
ou igual a $0$. Dessa forma, teremos descrito o controle e estado ótimos.

No caso numérico, podemos ter que analisar quando nossa função de troca vai ser 
maior, igual ou menor que zero. Porém, a igualdade a $0$ é complicada computacionalmente. 
Nesse sentido, estabelecemos um intervalo. No exemplo 17.4 do livro, quando o controle é 
Bang-Bang, houve convergência. Já o contrário não ocorreu. Como a função de troca é 
identicamente zero, problemas singulares são frequentemente instáveis. 

Pesquisa tem sido feita nesse sentido. A condição de Legendre-Clebsch é um exemplo. É uma
condição de segunda ordem, porque envolve ordem de derivadas mais altas.
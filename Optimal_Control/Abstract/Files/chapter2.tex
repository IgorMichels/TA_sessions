\textbf{Observação:} Se o funcional objetivo tiver valor mais ou menos infinito, dizemos que o problema não tem solução. Como assumimos a existência da solução, podemos obter um funcional que tem valor infinito, algo que não desejamos. 

\textbf{Teorema:} $J(u) := \int_{t_0}^{t_1} f(t,x(t),u(t))dt, s.a~x'(t) = g(t,x(t),u(t)), x(t_0) = x_0$. Ainda, $f, g \in C^1$ nos três argumentos e côncavos em $x$ e $u$. Sob as condições apresentadas anteriormente, adicionadas a $\la(t) \geq 0$, então para todo $u$, $J(u^*) \geq J(u)$. A demonstração é basicamente mostrar que a diferença é maior do que $0$.

\textbf{Teorema 2:} Seja $u \in L([t_0,t_1];\mathbb{R})$, $f$ é convexa em $u$ existem constantes $C_4$ e $C_1$,$C_2$,$C_3$ $>0$ e $\beta > 1$, tal que:
\begin{equation*}
    \begin{cases}
        g(t,x,u) = \alpha (t,x) + \beta (t,x)u \\
        |g(t,x,u)| \leq C_1(1 + |x| + |u|) \\
        |g(t,x_1,u) - g(t,x,u)| \leq C_2|x_1 - x|(1 + |u|) \\
        f(t,x,u) \geq C_3|u|^{\beta} - C_4
    \end{cases}
\end{equation*}
Então $J(u^*)$ é maximizador do funcional. Em problemas de minimização, $g$ seria côncava e a desigualdade de $f$ é revertida. 

Agora, temos que extender as condições necessárias para Lebesgue. 

\textbf{Unicidade:} Implica diretamente da unicidade das soluções do sistema de otimização (intervalos de tempo curto). A volta não é sempre verdadeira.  

\subsection{Interpretação da Adjunta}

Estabelecemos que $\frac{\partial V}{\partial x}(x_0, t_0) = \lim_{\e \to 0} \frac{V(x_0 + \e, t_0) - V(x_0,t_0)}{\e} = \la(t_0)$. Podemos relacionar, então, a função adjunta à variação marginal da função custo/lucro com respeito ao estado. É o valor adicional associado com um incremento adicional da variável de estado. Podemos aproximar: 

\begin{equation*}
    V(x_0 + \e,t_0) \approx V(x_0,t_0) + \e\la(t_0). 
\end{equation*}
Se $\e = 1$, podemos ver que ao adicionar um unidade de valor, $\la(t_0)$ é o valor objetivo adicional. 

\subsection{Princípio da Otimalidade}

\textbf{Teorema 3:} Considere $u^*$ o controle ótimo associado ao estado $x^*$ para o problema de já citado. Se $\hat{t}, t_1 < \hat{t} < t_1$, então as funções restritas ao intervalo $[\hat{t},t_1]$ formam uma solução ótima para o problema com tempo inicial $\hat{t}$. Além disso, será único, desde que $u^*$ seja. A demonstração ocorre por contradição. Note que nada pode ser dito sobre o intervalo $[t_0, \hat{t}]$, pois existem contra-exemplos. 

\textbf{Teorema 4:} $H(t,x,u,\la)$ é contínua por partes e contínua Lipschitz em relação ao tempo no caminho ótimo.

\textbf{Função Autônoma:} Quando não existe dependência do tempo nas funções $f$ e $g$. 

\textbf{Teorema 5:} Se um problema de controle ótimo é autônomo, então o Hamiltoniano é uma função que não depende do tempo. Note que se $M(t) := H(x^*(t),u^*(t), \la(t))$ é Lipschitz continua, sabemos que $M$ é diferenciável em quase toda parte com respeito à medida de Lebesgue. A partir disso, e utilizando o princípio máximo, vemos que $M'(t) = 0$ em quase toda parte. Como $M$ é contínua, ela é constante. 

\textbf{Princípio Máximo: } O máximo de uma função é encontrado em uma das bordas. 
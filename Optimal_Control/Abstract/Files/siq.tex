O isolamento/quarentena é um procedimento para controlar o avanço da doença.
Foi, provavelmente, um dos primeiros métodos de controle utilizados. Uma
epidemia é um surto de uma doença em um curto período de tempo. Inicialmente,
olharemos para os modelos SIR, com a adição de um novo compartimento, $Q$.
Assumimos, inicialmente, que aqueles em quarentena não infectam suscetíveis. 

Seja $S(t), I(t), Q(t) ~e~ R(t) $ o número de suscetíveis, infectados,
removidos e em quarentena no tempo $t$. Seja $\beta$ o número médio de
contatos suficientes para transmissão,  então $\beta\frac{I}{S + I + R}S$ é o
número de novos casos e é chamada de incidência padrão. Podemos aplicar a
lei das massas da mesma forma ($\beta SI$). 

\subsection{Simple mass action incidence}

Considere que haja imunidade permanente $R(t)$. O modelo: 

$$ S'(t) = A - \beta SI - dS,$$
$$ I'(t) = [\beta S - (\gamma + \delta + d + \alpha_1)]I,$$
$$ Q'(t) = \delta I - (d + \alpha_2 + \epsilon)Q,$$
$$ R'(t) = \gamma I + \epsilon Q - dR, $$

onde, $\alpha_1, \alpha_2$ representam os coeficientes de mortes causadas
pelas doeças. Afirmamos que $D = \{(S,I,Q,R) | S \geq 0, I \geq 0, Q \geq 0, R
\geq 0, S + I + Q + R \leq A/d\}$ é o conjunto das possíveis soluções, pois
$N(t) \to A/d$ quando não há doença. 

\subsubsection{Equilíbrio}

No equilíbrio, $I = 0$ ou $S = \frac{\gamma + \delta + d + \alpha_1}{\beta}$.
Se $I = 0$, $S = A/d, Q = 0, R = 0$. Se $I \neq 0, I(t) = \frac{A - dS}{\beta
S} = \frac{A - dS}{\gamma + \delta + d + \alpha_1}$. Nesse caso, $Q(t) =
\frac{\delta I}{d + \alpha_2 + \epsilon}$ e $R = \frac{\gamma I + \epsilon
Q}{d}$. 

Defina o número de reprodução de quarentena $R_q = \frac{\beta(A/d)}{\gamma +
\delta + d + \alpha_1}$, onde $\beta$ é a taxa de contato, e o denominador
indica o tempo médio de residência. Assim $R_q$ indica o número médio de
infecções secundárias em uma população completamente suscetível quando um
infeccioso entra na população. 

\subsection{Quarantine-adjusted incidence}

$$ S'(t) = A - \beta \frac{SI}{S + I + R} - dS,$$
$$ I'(t) = [\beta \frac{S}{S + I + R}  - (\gamma + \delta + d + \alpha_1)]I,$$
$$ Q'(t) = \delta I - (d + \alpha_2 + \epsilon)Q,$$
$$ R'(t) = \gamma I + \epsilon Q - dR, $$

\subsection{Modelo de Controle Proposto I}

$$\min_{u} \int_0^T CI(t) + [u(t)I(t)]^2 dt, $$
$$ s.a~ S'(t) = A - \beta\frac{SI}{S + I + R} - dS, $$
$$ I'(t) = \beta\frac{SI}{S + I + R} - (d + \alpha_1 + \gamma + u)I,  $$
$$ Q'(t) = uI - (\alpha_2 + d + \epsilon)Q,$$
$$ R'(t) = \gamma I + \epsilon Q - dR $$
$$ S(0) = S_0 \geq 0, I(0) = I_0  \geq 0, Q(0) = 0, R(t) = R_0 \geq 0,$$
$$ 0 \leq u(t) \leq 1 $$ 

\subsection{Modelo de Controle Proposto II - Controle Linear}

$$\min_{u} \int_0^T CI(t) + u(t)I(t) dt, $$
$$ s.a~ S'(t) = A - \beta\frac{SI}{S + I + R} - dS, $$
$$ I'(t) = \beta\frac{SI}{S + I + R} - (d + \alpha_1 + \gamma + u)I,  $$
$$ Q'(t) = uI - (\alpha_2 + d + \epsilon)Q,$$
$$ R'(t) = \gamma I + \epsilon Q - dR $$
$$ S(0) = S_0 \geq 0, I(0) = I_0  \geq 0, Q(0) = 0, R(t) = R_0 \geq 0,$$
$$ 0 \leq u(t) \leq 1 $$ 

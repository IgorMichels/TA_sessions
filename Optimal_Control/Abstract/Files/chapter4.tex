Queremos agora resolver os problemas de controle ótimo numericalmente. A equação $\frac{\partial H}{\partial u} = 0$ deve ser satisfeita em $u^*$ e pode ser de ajuda para encontrar $u$ em função de $x$ e $\la$. A partir disso, podemos utilizar um método como Runge-Kutta para resolver o sistema ótimo. Ele vai encontrar o controle ótimo se esse existir. 
\subsection{Algoritmo}
\begin{enumerate}
    \item Chute inicial para $\Vec{u}$, sendo cada coordenada de $u$ um valor no tempo discreto. 
    \item Resolva $x$ Foward utilizando a condição inicial e utilizando sua equação diferencial.
    \item Use a condição final de $\lambda$ e resolva Backward de acordo com sua equação diferencial.
    \item Atualize o vetor de controle. 
    \item Convergência. 
\end{enumerate}

É interessante utilizar uma combinação convexa entre o valor do controle anterior e o valor atual para acelerar a convergência.

\textbf{Combinação Convexa:}    Combinação Linear de pontos, cuja soma dos coeficientes é positiva e a soma é $1$.  

O erro no algoritmo é em geral o relativo e ele deve ser menor do que uma tolerância aceitável. A condição que obtemos é que $\delta \norm{\Vec{u}} - \norm{\Vec{u} - \Vec{oldu}} \geq 0$

\subsection{Runge-Kutta}
\begin{equation*}
    \begin{cases}
    x(t + h) \approx x(t) + \frac{h}{6}(k_1 + 2k_2 + 2k_3 + k_4) \\
    k1 = f(t,x(t)) \\
    k2 = f(t + \frac{h}{2},x(t) + \frac{h}{2}k_1) \\
    k3 = f(t + \frac{h}{2},x(t) + \frac{h}{2}k_2) \\
    k4 = f(t + h, x(t) + hk_3) \\
    \end{cases}
\end{equation*}
O erro é da ordem de $h^4$. 
Sabemos que, em geral, nossos problemas a serem resolvidos tem limitações no controle. Por exemplo, no uso de um químico, podemos indicar que o controle é necessariamente não negativo e tem, também, uma restrição legal, muitas vezes. 

\subsection{Condições Necessárias:} Esse problema será descrito da seguinte forma

\begin{align*}
    \max_u \int_{t_0}^{t_1} f(t,x(t),u(t)dt + \phi(x(t_1)) \\
    s.a.~x'(t) = g(t,x(t),u(t)), x(t_0) = x_0, \\
    a \leq u(t) \leq b, a < b
\end{align*}

Seja $u^*$ e $x^*$ o par ótimo. Observe que a derivada do funcional objetivo pode não ser zero no controle ótimo, pois $u^*$ pode estar nos limites do intervalo. Podemos avaliar o sinal da derivada, entretanto. Agora, dizemos que $\epsilon \in (0,\epsilon _0]$.  E, mais uma vez, reescrevemos o funcional e derivamos em relação a $\epsilon$, no ponto $0$, porém, nesse caso, essa derivada será menor ou igual a $0$. Tomando a função adjunta com as restrições já utilizadas anteriormente, reduzo a inequação para $0 \geq \int_{t_0}^{t_1} (f_u + \lambda g_u)h dt$, que vale para todos os valores de $h$. 

Seja $s$ um ponto de continuidade de $u^*$ com $a \leq u^*(s) < b$. Teremos que $f_u + \lambda g_u \leq 0$. Em contrapartida, se tivermos $a < u^*(s) \leq b$, concluiremos que $f_u + \lambda g_u \geq 0$. Os pontos que não há continuidade são irrelevantes. Sumariamente:

\begin{align*}
    u^*(t) = a \implies  f_u + \lambda g_u \leq 0 ~at~ t \\
    a < u^*(t) < b \implies f_u + \lambda g_u = 0 ~at~ t \\
    u^*(t) = b \implies f_u + \lambda g_u \geq 0 ~at~ t \\ 
    \iff \\
    f_u + \lambda g_u < 0 ~em~ t \implies u^*(t) = a \\
    f_u + \lambda g_u = 0 ~em~ t \implies a < u^*(t) < b \\
    f_u + \lambda g_u > 0 ~em~ t \implies u^*(t) = b \\
\end{align*}
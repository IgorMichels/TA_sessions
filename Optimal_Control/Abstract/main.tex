\documentclass[12pt]{article}
 
\usepackage[margin=1in]{geometry} 
\usepackage{amsmath,amsthm,amssymb}
\usepackage[margin=1in]{geometry} 
\usepackage{amsmath,amsthm,amssymb}
\usepackage[portuguese]{babel}
\usepackage[T1]{fontenc} 
\usepackage[utf8]{inputenc}
\usepackage{lmodern} 
\usepackage{graphicx}
\usepackage{setspace}
\graphicspath{ {images/} }
\usepackage{hyperref} 
\usepackage{mathtools}
\usepackage{commath}
\usepackage[sc,osf]{mathpazo}

\begin{document}


\newcommand{\e}{\epsilon}
\newcommand{\la}{\lambda}

\title{Resumo em Controle Ótimo}
\author{Lucas Machado Moschen\\ 
Escola de Matemática Aplicada}

\maketitle

\doublespacing
\section{Problemas Básicos de Controle Ótimo}
\subsection*{Exemplo 1}

Como motivação apresentada, consideramos duas equações que representam a variação do peso da parte vegetativa e reprodutiva, respectivamente. Nesse caso, o controle é a fração da fotossíntese destinada para a parte vegetativa. Nosso objetivo, nesse caso, é maximixar: 
\begin{equation*}
F(x,u,t) := \int_0^T \ln(x_2(t))dt, s.a. 
\end{equation*} 
\begin{equation*}
\begin{cases}
x_1'(t) = u(t)x_1(t) \\
x_2'(t) = (1 - u(t))x_2(t) \\
0 \leq u(t) \leq 1
\end{cases}    
\end{equation*}

Desta maneira, em um sistema de controle há variáveis de estado, funções de controle, que afetam a dinêmica do sistema, e o funcional objetivo, que funciona como o cálculo do lucro. 

\subsection{Definições Importantes}
\begin{enumerate}
    \item \textbf{Continuidade por partes:} Se uma função é contínua em cada ponto em que é definida, exceto em uma quantidade finita e é igual a seu limite à esquerda ou à direita me cada ponto (não permite pontos deslocados). 
    \item \textbf{Diferenciável por partes:} Função contínua que é diferenciável em cada ponto que é definida, exceto uma finidade.
    \item \textbf{Côncava: } Se $\forall 0 \leq \alpha \leq 1$ e $\forall a \leq t_1,t_2 \leq b$, $\alpha k(t_1) + (1 - \alpha)k(t_2) \leq k(\alpha t_1 + (1 - \alpha)t_2)$.
    \item \textbf{Lipschitz: } $|k(t_1) - k(t_2)| \leq c|t_1 - t_2|$.
    \item \textbf{Teorema do Valor Médio}
    \item \textbf{Teorema da Convergência Dominante: } Considere uma sequência $\{f_n\}$ dominada por uma função Lebesgue integrável $g$. Suponha que sequência converge ponto a ponto para uma função $f$. Então $f$ é integrável e $\lim_{n \to \infty} \int_S f_n d\mu = \int_S f d\mu$.
\end{enumerate}

\subsection*{Exercício 1}
Se $k: I \to \mathbb{R}$ é diferenciável por partes em um intervalo limitado, $k$ é Lipschitz. 

\textbf{Prova:} Seja $P = \{p_0, ..., p_n\}$ pontos em que $k$ não é diferenciável. Pelo teorema do valor médio, $\forall i, \exists x_i^0 $, tal que $k(p_i) - k(p_{i-1}) = k'(x_i^0)(p_i - p_{i-1})$. Como $k'$ é contínua em cada intervalo $[p_{i-1},p_i]$, basta tomar o valor máximo da derivada nesse intervalo. Depois disso, basta tomar o maior valor entre todas as derivadas e a desigualdade de Lipschitz é satisfeita. 

\subsection{Problema} 

Considere $x'(t) = g(t,x(t),u(t)) \to u(t) \mapsto x(u)$. Queremos, então, dado um funcional $ J(u) := \int_{t_0}^{t_1} (f(t,x(t),u(t))dt, x(t_0) = x_0$, maximizá-lo.  $u(t)$ é contínua por partes. 

\textbf{Função Adjunta:} proposta similar aos multiplicadores de Lagrange. $\lambda : [t_0,t_1] \to \mathbb{R}$ é diferenciável por partes e deve satisfazer algumas condições. 

Para isso, assumimos a existência $u^*$ e $x^*$. Nesse caso, $J(u) \leq J(u^*) < \infty$. Tome $u^{\epsilon} = u^{*}(t) + \epsilon h(t), x^{\epsilon}(t_0) = x_0 \implies \lim_{\epsilon\to 0} u^{\epsilon} = u^{*}$. $x^{\epsilon}$ é o estado associado ao controle. Como a função $g$ é continuamente diferenciável, $x^{\epsilon} \to x^{*}$. Assim, a sua derivada em $\epsilon = 0$ existe. Utilizo, pelo TFC, que $\int_{t_0}^{t_1} \frac{d}{dt}(\lambda (t)x^{\epsilon}(t))dt = \lambda(t_1)x^{\epsilon}(t_1) - \lambda(t_0)x^{\epsilon}(t_0)$. Desta maneira, utilizo que $J(u^{\epsilon}) = \int_{t_0}^{t_1} f(t,x^{\epsilon}(t),u^{\epsilon}(t))dt - \lambda(t_1)x^{\epsilon}(t_1) + \lambda(t_0)x^{\epsilon}(t_0)$. Sabemos que $lim_{\epsilon \to 0} \frac{J(u^{\epsilon}) - J(u^{*})}{\epsilon} = 0$, pois $J(u^*)$ é máximo. Desta maneira, 
\begin{equation} \label{eq1}
0 = \frac{d}{d\epsilon} J(u^{\epsilon})_{\e = 0} = \int_{t_0}^{t_1} [(f_x + \lambda(t)g_x + \lambda '(t))\frac{\partial x^{\e}}{\partial\e}(t)_{\e = 0} + (f_u + \lambda(t)g_u)h(t)]dt - \lambda(t_1)\frac{\partial x^{\e}}{\e}(t_1)_{\e = 0}
\end{equation}
Note que isso pode ser feito pelo Teorema da Convergência Dominante, pois podemos mover o limite (derivada) para dentro da integral (intervalo compacto e integrando é diferenciável po partes).  Para que ocorra a igualdade citada acima, definimos $H(t,x,u,\lambda) := f(t,x,u) + \lambda g(t,x,u)$ e:
\begin{equation}
\begin{cases}
    \frac{\partial H}{\partial u}_{u = u^{*}} = 0 \\
    \frac{\partial H}{\partial x}_{x = x^*} = - \lambda ' \\
    \frac{\partial H}{\partial \lambda} = x' \\
    \lambda(t_1) = 0 
\end{cases}    
\end{equation}

\subsection{Princípio Máximo de Pontryagin}

Se $u^*$ e $x^*$ são controle ótimo, então existe $\lambda (t)$ diferenciável por partes tal que a função $H$, como definida anteriormente, pode ser maximizada em $u^*(t)$. A demonstração é mais simples para o caso de $f$ e $g$ côncavas em $u$ e $\lambda (t) \geq 0$. A segunda derivada do Hamiltoniano indica o tipo de problema: Se for negativa, é um problema de maximização. 
\textbf{Observação:} A condição de maximizar $H$ não sempre implica que $H_u = 0$.  

\subsection{Exercício 1.6 - Efeito Alle}

Nesse efeito, consideramos um valor mínimo. O crescimento $x'(t) = rx(t)(\frac{x(t)}{x_{min}} - 1)(1 - \frac{x(t)}{x_{max}})$



\section{Existência e Outras Propriedades}
\textbf{Observação:} Se o funcional objetivo tiver valor mais ou menos infinito, dizemos que o problema não tem solução. Como assumimos a existência da solução, podemos obter um funcional que tem valor infinito, algo que não desejamos. 

\textbf{Teorema:} $J(u) := \int_{t_0}^{t_1} f(t,x(t),u(t))dt, s.a~x'(t) = g(t,x(t),u(t)), x(t_0) = x_0$. Ainda, $f, g \in C^1$ nos três argumentos e côncavos em $x$ e $u$. Sob as condições apresentadas anteriormente, adicionadas a $\la(t) \geq 0$, então para todo $u$, $J(u^*) \geq J(u)$. A demonstração é basicamente mostrar que a diferença é maior do que $0$.

\textbf{Teorema 2:} Seja $u \in L([t_0,t_1];\mathbb{R})$, $f$ é convexa em $u$ existem constantes $C_4$ e $C_1$,$C_2$,$C_3$ $>0$ e $\beta > 1$, tal que:
\begin{equation*}
    \begin{cases}
        g(t,x,u) = \alpha (t,x) + \beta (t,x)u \\
        |g(t,x,u)| \leq C_1(1 + |x| + |u|) \\
        |g(t,x_1,u) - g(t,x,u)| \leq C_2|x_1 - x|(1 + |u|) \\
        f(t,x,u) \geq C_3|u|^{\beta} - C_4
    \end{cases}
\end{equation*}
Então $J(u^*)$ é maximizador do funcional. Em problemas de minimização, $g$ seria côncava e a desigualdade de $f$ é revertida. 

Agora, temos que extender as condições necessárias para Lebesgue. 

\textbf{Unicidade:} Implica diretamente da unicidade das soluções do sistema de otimização (intervalos de tempo curto). A volta não é sempre verdadeira.  

\subsection{Interpretação da Adjunta}

Considere o funcional $V(x_0, t_0)$ a ser maximizado. Estabelecemos que $\frac{\partial V}{\partial x}(x_0, t_0) = \lim_{\e \to 0} \frac{V(x_0 + \e, t_0) - V(x_0,t_0)}{\e} = \la(t_0)$. Podemos relacionar, então, a função adjunta à variação marginal da função custo/lucro com respeito ao estado. É o valor adicional associado com um incremento adicional da variável de estado. Podemos aproximar: 

\begin{equation*}
    V(x_0 + \e,t_0) \approx V(x_0,t_0) + \e\la(t_0). 
\end{equation*}
Se $\e = 1$, podemos ver que ao adicionar um unidade de valor, $\la(t_0)$ é o valor objetivo adicional. 

\subsection{Princípio da Otimalidade}

\textbf{Teorema 3:} Considere $u^*$ o controle ótimo associado ao estado $x^*$ para o problema de já citado. Se $\hat{t}, t_1 < \hat{t} < t_1$, então as funções restritas ao intervalo $[\hat{t},t_1]$ formam uma solução ótima para o problema com tempo inicial $\hat{t}$. Além disso, será único, desde que $u^*$ seja. A demonstração ocorre por contradição. Note que nada pode ser dito sobre o intervalo $[t_0, \hat{t}]$, pois existem contra-exemplos. 

\textbf{Teorema 4:} $H(t,x,u,\la)$ é contínua por partes e contínua Lipschitz em relação ao tempo no caminho ótimo.

\textbf{Função Autônoma:} Quando não existe dependência do tempo nas funções $f$ e $g$. 

\textbf{Teorema 5:} Se um problema de controle ótimo é autônomo, então o Hamiltoniano é uma função que não depende do tempo. Note que se $M(t) := H(x^*(t),u^*(t), \la(t))$ é Lipschitz continua, sabemos que $M$ é diferenciável em quase toda parte com respeito à medida de Lebesgue. A partir disso, e utilizando o princípio máximo, vemos que $M'(t) = 0$ em quase toda parte. Como $M$ é contínua, ela é constante. 

\textbf{Princípio Máximo: } O máximo de uma função é encontrado em uma das bordas. 

\section{Condições finais}
\subsection{Termo Payoff}


Muitas vezes, também queremos maxmizar o valor de uma função em um determinado tempo, em especial no final do intervalo. Nesse caso, o problema se torna:
\begin{equation*}
    \begin{cases}
    \max_u [\phi(x(t_1)) + \int_{t_0}^{t_1} f(t, x(t),u(t))dt] \\
    x' = g(t,x(t),u(t)), x(t_0) = x_0
    \end{cases}
\end{equation*}

A função $\phi$ é conhecida como termo payoff. A única mudança na obtenção das condições necessárias é na condição do tempo final. Obtemos que $\la(t_1) = \phi '(x^*(t_1))$. (Mais uma vez, precisamos fazer com que $\lim_{\e \to 0} \frac{J(u^{\e}) - J(u^*)}{\e} = 0$

\subsection{Estados com Pontos Finais Fixados}

\textbf{Obs.:} O funcional objetivo ser imaterial significa que não depende da condição final do estado. 

Podemos deixar $x(t_0)$ livre e $x(t_1) = x_1$ fixado. Essa caso é similar com o anterior, com a mudança de que $\la(t_0) = \phi '(x(t_0))$. Isso sugere que exista uma dualidade entre as condições de estado e adjunta.

Também podemos fixar os pontos inicial e final de estado. Notamos que estamos considerando a maximização sobre o conjunto de controles admissíveis, que respeitem as condições, inclusive sobre a variável de estado. 

\textbf{Teorema 1:} Se $u^*(t)$ e $x^*(t)$ são ótimos para o problema com pontos inicial e final fixados, então existe uma função $\la(t)$ diferenciável por partes e uma contante $\la_0$ igual a $0$ ou $1$, onde $H = \la_0f(t,x(t),u(t)) + \la(t)g(t,x(t),u(t))$ e $\la '(t) = - H_x$. 

A diferença é que a função adjunta não tem restrições. A demonstração utiliza uma técnica diferente da utilizada até então. A constante ajusta para problemas degenerados ou problema tem funcional objetivo imaterial.

\section{Método Backward/Foward}
Queremos agora resolver os problemas de controle ótimo numericalmente. A equação $\frac{\partial H}{\partial u} = 0$ deve ser satisfeita em $u^*$ e pode ser de ajuda para encontrar $u$ em função de $x$ e $\la$. A partir disso, podemos utilizar um método como Runge-Kutta para resolver o sistema ótimo. Ele vai encontrar o controle ótimo se esse existir. 
\subsection{Algoritmo}
\begin{enumerate}
    \item Chute inicial para $\Vec{u}$, sendo cada coordenada de $u$ um valor no tempo discreto. 
    \item Resolva $x$ Foward utilizando a condição inicial e utilizando sua equação diferencial.
    \item Use a condição final de $\lambda$ e resolva Backward de acordo com sua equação diferencial.
    \item Atualize o vetor de controle. 
    \item Convergência. 
\end{enumerate}

É interessante utilizar uma combinação convexa entre o valor do controle anterior e o valor atual para acelerar a convergência.

\textbf{Combinação Convexa:}    Combinação Linear de pontos, cuja soma dos coeficientes é positiva e a soma é $1$.  

O erro no algoritmo é em geral o relativo e ele deve ser menor do que uma tolerância aceitável. A condição que obtemos é que $\delta \norm{\Vec{u}} - \norm{\Vec{u} - \Vec{oldu}} \geq 0$

\subsection{Runge-Kutta}
\begin{equation*}
    \begin{cases}
    x(t + h) \approx x(t) + \frac{h}{6}(k_1 + 2k_2 + 2k_3 + k_4) \\
    k1 = f(t,x(t)) \\
    k2 = f(t + \frac{h}{2},x(t) + \frac{h}{2}k_1) \\
    k3 = f(t + \frac{h}{2},x(t) + \frac{h}{2}k_2) \\
    k4 = f(t + h, x(t) + hk_3) \\
    \end{cases}
\end{equation*}
O erro é da ordem de $h^4$. 

\section{Laboratórios}
\subsection{Laboratório 1}

Nesse laboratório, o autor explora a utilização do MatLab como ferramenta, devido à facilidade de se trabalhar com essa linguagem matematicamente e pela quanlidade grafica dos resultados. 

Além disso, ele resolve um problema de controle ótimo. 

\subsection{Laboratório 2}

Aplicação em Biologia. Dada uma população com capacidade máxima (carrying capacity), queremos reduzí-la. Nesse caso, o controle é quantidade adicionada no tempo $t$. Assim, o problema se reduz a:
\begin{equation}
    \begin{cases}
    \min_u \int_0^T (Ax(t)^2 + u(t)^2) dt \\
    s.a.~~ x'(t) = r(M - x(t)) - u(t)x(t), x(0) = x_0
    \end{cases}
\end{equation}

\subsection{Laboratório 3}

Aplicação sobre Bactéria. Nesse laboratório, o tópico pe sobre o crescimento de uma bactéria quando um nutriente qímico é utilizado para acelerar a reprodução. Nosso problema, então:
\begin{equation}
    \begin{cases}
    \max_u Cx(1) - \int_0^1 u(t)^2 dt \\
    s.a.~~ x'(t) = rx(t) + Au(t)x(t) - Bu(t)^2e^{-x(t)}, \\
    x(0) = x_0, A, B, C \geq 0
    \end{cases}
\end{equation}

Como $\la(t) > 0 \forall t$, podemos obter a caracterização do controle como comumente fazemos.



\section{Controles Limitados}
Sabemos que, em geral, nossos problemas a serem resolvidos tem limitações no controle. Por exemplo, no uso de um químico, podemos indicar que o controle é necessariamente não negativo e tem, também, uma restrição legal, muitas vezes. 

\subsection{Condições Necessárias:} Esse problema será descrito da seguinte forma

\begin{align*}
    \max_u \int_{t_0}^{t_1} f(t,x(t),u(t)dt + \phi(x(t_1)) \\
    s.a.~x'(t) = g(t,x(t),u(t)), x(t_0) = x_0, \\
    a \leq u(t) \leq b, a < b
\end{align*}

Seja $u^*$ e $x^*$ o par ótimo. Observe que a derivada do funcional objetivo pode não ser zero no controle ótimo, ppois $u^*$ pode estar nos limites do intervalo. Podemos avaliar o sinal da derivada, entretanto. Agora, dizemos que $\epsilon \in (0,\epsilon _0]$.  E, mais uma vez, reescrevemos o funcional e derivamos em relação a $\epsilon$, no ponto $0$, porém, nesse caso, essa derivada será menor ou igual a $0$. Tomando a função adjunta com as restrições já utilizadas anteriormente, reduzo a inequação para $0 \geq \int_{t_0}^{t_1} (f_u + \lambda g_u)h dt$, que vale para todos os valores de $h$. 

Seja $s$ um ponto de continuidade de $u^*$ com $a \leq u^*(s) < b$. Teremos que $f_u + \lambda g_u \leq 0$. Em contrapartida, se tivermos $a < u^*(s) \leq b$, concluiremos que $f_u + \lambda g_u \geq 0$. Os pontos que não há continuidade são irrelevantes. Sumariamente:

\begin{align*}
    u^*(t) = a \implies  f_u + \lambda g_u \leq 0 ~at~ t \\
    a < u^*(t) < b \implies f_u + \lambda g_u = 0 ~at~ t \\
    u^*(t) = b \implies f_u + \lambda g_u \geq 0 ~at~ t \\ 
    \iff \\
    f_u + \lambda g_u < 0 ~at~ \implies u^*(t) = a \\
    f_u + \lambda g_u = 0 ~at~ t \implies a < u^*(t) < b \\
    f_u + \lambda g_u > 0 ~at~ t \implies u^*(t) = b \\
\end{align*}

\section{Laboratórios}
\subsection{Laboratório 4}

É um reexame do primeiro laboratório. A primeira análise é de como o controle muda quando há uma restrição (o que faz sentido). 

\subsection{Laboratório 5 - Cancer}

Queremos minimizar a densidade do tumor e os efeitos colaterais das drogas. É assumido que o tumor tenha um crescimento Gompertzian. O modelo utilizado no laboratório é Skiper's log-kill hipótese, que afirma que a morte de células devido às drofas é proporcional a população de tumor. 

Considere $N(t)$ a densidade normalizada do tumor no tempo $t$. Assim:
$$N'(t) = rN(t)\ln(\frac{1}{N(t)} - u(t)\delta N(t)$$
$r$ é a taxa de crescimento do tumor, $\delta$ a magnitude da dose e $u(t)$ descreve a ação da droga. É a força do efeito da droga. Escolhemos o funcional para ser 
$$\min_u \int_0^T aN(t)^2 + u(t)$$

Além disso, $u(t) \geq 0$ e $N(0) = N_0$. 

\subsection{Laboratório 6 - Fish Harvesting}

Suponha que em um tanque em $t = 0$ são adicionados peixes com massa média essencialmente 0. Além, descrevemos a massa do peixe segundo $f(t) = \frac{kt}{t+1}$. Note que $\lim f(t) = k$. Considere um intervalo $[0,T]$, com $T$ pequeno suficiente para que não haja reprodução. Queremos:

$$
\max_u \int_0^T A\frac{kt}{t+1}x(t)u(t) - u(t)^2 dt
$$
$$
subject~to~x'(t) = -(m + u(t))x(t), x(0) = x_0, 0 \leq u(t) \leq M
$$
$M$ é um limite físico para a taxa de colheita. Note que para qualquer valor de $u(t) > 0$, a tax avai decrescer. 

Como nos laboratórios anteriores, o valor T influencia o controle ótimo. 



\section{Optimal Control of Several Variables}
Agora o problema se resume a:

$$
max_{u_1,...,u_m} \int_{t_0}^{t_1} f(t,x_1(t),...,x_n(t),u_1(t),...,u_m(t)) dt + \phi(x_1(t_1),...,x_n(t_1))
$$
$$
subject~to~x_i'(t) = g_i(t,x_1(t),...,x_n(t),u_1(t),...,u_m(t))
$$
$$
x_i(t_0) = x_{i0}~for~i=1,2,...,n
$$
onde as função $f$ e $g_i$ são continuamente diferenciáveis em cada variável. Podemos usar a expressão em forma de vetores. Considere $\Vec{\la}(t) = [\la_1(t),...,\la_n(t)]$ um vetor com funções diferenciáveis por partes. Definimos $H(t,\Vec{x},\Vec{u},\Vec{\la}) = f(t,\Vec{x},\Vec{u}) + \Vec{\la}(t)\cdot \Vec{g}(t,\Vec{x},\Vec{u})$. Se fizermos o mesmo processo anterior, vamos obter as condições:
$$
x'_i(t) = \frac{\partial H}{\partial \la_i} = g_i(t,\Vec{x},\Vec(u)), x_i(0) = x_{i0}~for~i = 1,...n 
$$
$$
\la'_j(t) = - \frac{\partial H}{\partial x_j}, \la_j(t_1) = \phi_{x_j}(\Vec{x}(t_1))~for~j=1,...,n
$$
$$
0 = \frac{\partial H}{\partial u_k}~at~u^*_k~for~k=1,...,m
$$
Outros ajustes vistos nos capítulos anteriores ocorrem de mesma forma no caso multivariado. Inclusive se os limites das variáveis de controle estiverem presentes, o que altera as condições, também. 

\subsection{Problemas Linear Quadratic Regulator}

Considere a equação diferencial do estado linear em $x$ e $u$ e o funcional objetivo na forma quadrática. 
$$
J(u) := \frac{1}{2}[x^T(T)Mx(t) + \int_0^T x^T(t)Q(t)x(t) + u^TR(t)u(t) dt]
$$
$$
x'(t) = A(t)x(t) + B(t)u(t)
$$
Onde $M, Q(t)$ são positivas semidefinidas e $R(t)$ é positiva definida para garantir invertibilidade. As três são simétricas. Observe que isso garante a diagonalização.
Assim: $H = \frac{1}{2}x^TQx + \frac{1}{2}u^TRu + \la^T(Ax + Bu).$ 

Deste modo $u^* = - R^{-1}B^T\la$ e $\la' = -Qx - A^T\la, \la(T) = Mx(T)$. Se supormos que $\la = Sx$, chegamos que $S'x + Sx' = -Qx - A^T\la$. Com as devidas transformações. Obtermos que $-S'x = Qx + A^TSx - SBR^{-1}B^TSx$, com $S(T) = M$. Isso nos mostra que equação matricial Ricatti, que $S(t)$ deve satisfazer. Basta resolver o problema. Por fim $u^* = -R^{-1}B^TSx$. Essa matriz é chamada de ganho. 

\subsection{Equações Diferenciais de Ordem mais Alta}
Quando temos uma equação diferencial de ordem mais alta, podemos definir um sistema com as derivadas, onde $x_1(t) = x(t), x_2(t) = x'(t), ..., x_{n+1}(t) = x^{(n)}(t)$. A partir disso, podemos resolver com o Princípio Máximo de Pontryagin. 

\subsection{Limites Isoperimétricos}

Além dos limites inferior e superior que podemos colocar no controle, também podemos querer que o exista limites na integral do controle. Exemplo: Para medicar uma pessoa, podemos querer que a quantidade total de remédia seja um valor $B$. 
Assim, a restrição é do tipo $\int_0^T u(t) dt = B$. De forma geral, podemos ter $\int_{t_0}^{t_1} h(t,x(t),u(t)) dt = B$, sendo $h$ continuamente diferenciável, como restrição. Desta maneira, não podemos usar o Princípio Máximo de Pontryagin. Para isso, introduzimos $z(t) = \int_{t_0}^t h(s,x(s),u(s))ds$. Desta maneira, nosso problema terá duas variáveis de estado. 

\subsection{Soluções Numéricas}

Agora, para cada controle, um valor inicial para o controle é dado. Depois as variáveis de estado são resolvidas simultaneamente. Por fim, as adjuntas. 

\section{Linear Dependence on the Control}
Vamos considerar problemas especiais em que o problema é linear 
no controle $u(t)$. 

\subsection{Controle Bang-Bang}

Considere o problema de controle ótimo.

$$max_u \int_{t_0}^{t_1} f_1(t,x) + u(t)f_2(t,x) dt$$
$$s.a.~~x'(t) = g_1(t,x) + u(t)g_2(t,x), x(0) = x_0$$
$$ a \leq u(t) \leq b$$

Assim $H(t,x,u,\la) = f_1(t,x) + \la g_1(t,x) + u(t)(f_2(t,x) + \la g_2(t,x))$, 
linear em $u(t)$. Observe a derivada parcial em relação a $u$ não
carrega informação sobre $u(t)$. Assim definimos $\psi(t) := f_2(t,x(t)) + \la(t)g_2(t,x(t))$,
muitas vezes chamada de função de troca. Se $\phi = 0$ não pode ser obtido 
em um intervalo de tempo, mas ocorre apenas em pontos finitos, o controle
é dito "Bing Bang", porque só varia entre os valores mínimo e máximo de $u(t)$. 
Os valores de $u(t)$ nesses pontos não são de interesse, portanto. 

Em contrapartida, se $\psi(t) \equiv 0$ em um intervalo de tempo, dizemos que $u^*$ é 
singular nesse intervalo. Esse caso será explorado na próxima sessão. 

Para resolver esse tipo de problema, primeiro precisamos verificar se de fato 
o problema é Bang-Bang. Numericamente, o problema é apenas em verificar a positividade
ou negatividade da função de troca. 

\subsection{Controles Singulares}

O livro explora um exemplo inicial:

$$max_u \int_0^2 (x(t) - t**2)**2 dt $$
$$s.a. ~~ x'(t) = u(t), x(0) = 1$$
$$0 \leq u(t) \leq 4$$

Calculamos o Hamiltoniano e encontramos $u^*(t)$ em função da adjunta. 
Para sair dessa hipótese, precisamos fazer uma análise mais minunciosa. 
Ela começa em provar que $x(t) \geq t**2 \rightarrow \lambda'(t) \leq 0 \wedge \la(t) \geq 0$. 
Então, basta encontrar os valores de $t$ em que essa função é positiva 
ou igual a $0$. Dessa forma, teremos descrito o controle e estado ótimos.

\section{SIQ models}
O isolamento/quarentena é um procedimento para controlar o avanço da doença.
Foi, provavelmente, um dos primeiros métodos de controle utilizados. Uma
epidemia é um surto de uma doença em um curto período de tempo. Inicialmente,
olharemos para os modelos SIR, com a adição de um novo compartimento, $Q$.
Assumimos, inicialmente, que aqueles em quarentena não infectam suscetíveis. 

Seja $S(t), I(t), Q(t) ~e~ R(t) $ o número de suscetíveis, infectados,
removidos e em quarentena no tempo $t$. Seja $\beta$ o número médio de
contatos suficientes para transmissão,  então $\beta\frac{I}{S + I + R}S$ é o
número de novos casos e é chamada de incidência padrão. Podemos aplicar a
lei das massas da mesma forma ($\beta SI$). 

\subsection{Simple mass action incidence}

Considere que haja imunidade permanente $R(t)$. O modelo: 

$$ S'(t) = A - \beta SI - dS,$$
$$ I'(t) = [\beta S - (\gamma + \delta + d + \alpha_1)]I,$$
$$ Q'(t) = \delta I - (d + \alpha_2 + \epsilon)Q,$$
$$ R'(t) = \gamma I + \epsilon Q - dR, $$

onde, $\alpha_1, \alpha_2$ representam os coeficientes de mortes causadas
pelas doeças. Afirmamos que $D = \{(S,I,Q,R) | S \geq 0, I \geq 0, Q \geq 0, R
\geq 0, S + I + Q + R \leq A/d\}$ é o conjunto das possíveis soluções, pois
$N(t) \to A/d$ quando não há doença. 

\subsubsection{Equilíbrio}

No equilíbrio, $I = 0$ ou $S = \frac{\gamma + \delta + d + \alpha_1}{\beta}$.
Se $I = 0$, $S = A/d, Q = 0, R = 0$. Se $I \neq 0, I(t) = \frac{A - dS}{\beta
S} = \frac{A - dS}{\gamma + \delta + d + \alpha_1}$. Nesse caso, $Q(t) =
\frac{\delta I}{d + \alpha_2 + \epsilon}$ e $R = \frac{\gamma I + \epsilon
Q}{d}$. 

Defina o número de reprodução de quarentena $R_q = \frac{\beta(A/d)}{\gamma +
\delta + d + \alpha_1}$, onde $\beta$ é a taxa de contato, e o denominador
indica o tempo médio de residência. Assim $R_q$ indica o número médio de
infecções secundárias em uma população completamente suscetível quando um
infeccioso entra na população. 

\subsection{Quarantine-adjusted incidence}

$$ S'(t) = A - \beta \frac{SI}{S + I + R} - dS,$$
$$ I'(t) = [\beta \frac{S}{S + I + R}  - (\gamma + \delta + d + \alpha_1)]I,$$
$$ Q'(t) = \delta I - (d + \alpha_2 + \epsilon)Q,$$
$$ R'(t) = \gamma I + \epsilon Q - dR, $$

\subsection{Modelo de Controle Proposto I}

$$\min_{u} \int_0^T CI(t) + [u(t)I(t)]^2 dt, $$
$$ s.a~ S'(t) = A - \beta\frac{SI}{S + I + R} - dS, $$
$$ I'(t) = \beta\frac{SI}{S + I + R} - (d + \alpha_1 + \gamma + u)I,  $$
$$ Q'(t) = uI - (\alpha_2 + d + \epsilon)Q,$$
$$ R'(t) = \gamma I + \epsilon Q - dR $$
$$ S(0) = S_0 \geq 0, I(0) = I_0  \geq 0, Q(0) = 0, R(t) = R_0 \geq 0,$$
$$ 0 \leq u(t) \leq 1 $$ 

\subsection{Modelo de Controle Proposto II - Controle Linear}

$$\min_{u} \int_0^T CI(t) + u(t)I(t) dt, $$
$$ s.a~ S'(t) = A - \beta\frac{SI}{S + I + R} - dS, $$
$$ I'(t) = \beta\frac{SI}{S + I + R} - (d + \alpha_1 + \gamma + u)I,  $$
$$ Q'(t) = uI - (\alpha_2 + d + \epsilon)Q,$$
$$ R'(t) = \gamma I + \epsilon Q - dR $$
$$ S(0) = S_0 \geq 0, I(0) = I_0  \geq 0, Q(0) = 0, R(t) = R_0 \geq 0,$$
$$ 0 \leq u(t) \leq 1 $$ 


\end{document}
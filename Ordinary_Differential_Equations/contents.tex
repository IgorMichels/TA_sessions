\title{Conteúdos da Matéria Equações Diferencias Ordinárias}
\author{Lucas Moschen \\ 
        Fundação Getulio Vargas}
\date{\today}

\documentclass[12pt]{article}

\usepackage[portuguese]{babel}
\usepackage{amsmath}
\usepackage{physics}

\begin{document}
\maketitle

\begin{abstract}
    Neste documento irei constar os principais temas cobertos pela matéria,
    que tem foco em um cálculo de edos, sem grandes definições precisas e
    estudo do comportamento qualitativo. Qualquer correção nesse documento
    pode ser sugerida pelo leitor através de um \textit{pull request}. Para
    iniciar, irei listar os temas até agora cobertos e também inserirei um
    pequeno resumo sobre o determinado tópico. 
\end{abstract}

\tableofcontents

\section{Equações Diferenciais Lineares de Primeira Ordem}

Formato: $\dv{y}{x} + p(x)y = q(x)$. Observe a linearidade de $y$ e que a sua
derivada de maior ordem é a primeira. Para resolver esse exemplo, usamos oo
fator de integração $u(x) = e^{\int p(x) dx}$ e multiplicamos em ambos os
lados. Observe que escolhemos ele, porque queremos $(y\cdot u)' = y'\cdot u +
y\cdot u' = u\cdot q$ e $u' = u\cdot p$. A partir disso, obstemos que
$y(x)u(x) = \int u(x)q(x) dx$. 

\subsection{Equações de Bernoulli}

Formato: $y' + p(x)y = q(x)y^n$. Neste caso temos que o expoente de $y$ é de
ordem $n$. Para resolver esse problema, supomos que $y \neq 0$ e fazemos uma
transformação de variável $z(x) = [y(x)]^{1-n}, \forall x$. Essa transformação
vai noos permitir obter a equação em um formato desejado. Para ver isso,
primeiro façamos $\dv{z}{x} = \dv{z}{y}\dv{y}{x} =
(1-n)y^{-n}\dv{y}{x}$, logo, substituindo os valores, teremos que
$\frac{1}{1-n}y^{n}z' + p(x)zy^n = q(x)y^n \implies z' + (1-n)p(x)z = q(x)$ e
resolvemos pelo formato anterior. 

\section{Equações com Variáveis Separáveis}

Formato: $\dv{y}{x} = f(x,y) = \phi(x)\psi(y)$, isto é, a derivada pode ser
escrita como um produto de uma função que só depende de $x$ por outra que só
depende de $y$. Nesse caso, usamos a reescrita diferencial para poder escrever
isso da seguinte forma: $\int \frac{dy}{\psi(y)} = \int \phi(x)dx$. Isso pode
ser extendido quando a função pode ser escrita como uma divisão de funções
desse tipo, bastando vê-la como um produto.  

\section{Equações Exatas}

Formato: Seja $\dv{dy}{dx} = f(x,y) = - \frac{M(x,y)}{N(x,y)}$ que pode ser
reescrita da forma $M(x,y)dx + N(x,y)dy = 0$. Ela é caracterizada como
\textbf{exata} se $\exists g(x,y)$, tal que $dg = Mdx + Ndy$, onde $dg$ é o
diferencial de $g$. Isto é, $\pdv{g}{x} = M$ e $\pdv{g}{y} = N$. Nesse caso,
podemos provar pelo teorema de Clairaut-Schwars que $\pdv{M(x,y)}{y} =
\pdv{N(x,y)}{x}$ (*). 

\subsection{Fator de Integração}

Suponha que a equação $M(x,y)dx + N(x,y)dy = 0$ seja não exata. Nesse caso, a
ideia é encontrar uma função $u$ que ao multiplicar a equação, obtenha-se a
hipótese do teorema de Clairaut-Schwars, como mencionado acima (*). Nesse
caso, se $\frac{M_y - N_x}{N}$ é função apenas de $x$, o fator de integração
será $u(x) = \exp{\int \frac{M_y - N_x}{N}dx}$. Para construir esse resultado,
basta pensar, supondo a existência de $u(x)$, temos que $\pdv{(u\cdot M)}{y} =
u\pdv{M}{y} = \pdv{(u\cdot N)}{x} = \dv{u}{x}N + u\pdv{N}{x}$. Agora, se $\frac{N_x
- M_y}{M}$ é função apenas de $y$, vale que $u(y) = \exp{\int \frac{N_x -
M_y}{M}dy}$. 

\section{Modelos da Dinâmica de uma População}

\subsection{Malthus} 
Também conhecido como modelo exponencial, é
baseado na ideia de que o crescimento populacional é proporcional ao
tamanho da população, o que faz um certo sentido. O modelo é parte da
ideia de que existiria um ponto em que o número de pessoas seria maior do
que o suporte para a alimentação que tem crescimento linear. Nesse caso,
se $p(t)$ é a população no tempo $t$, o crescimento é dado por $p'(t) =
rp(t)$. Esse coeficiente $r$ vai indicar a taxa de crescimento
populacional, e ele é tratado como constante. Essa ideia foi descartada
posteriormente, pois o crescimento reduziu suas taxas de crescimento desde
os anos de 1800. Nesse caso, $p(t) = p(0)e^{rt}$. 

\subsection{Logística de Verhuslt} 

Também conhecido como curva S o função logística. Diferente do primeiro
modelo, ele não assume que os
recursos são ilimitados. Entretanto, ele assume a existência da capacidade
de carga $K$, que é o tamanho populacional máximo que o meio pode
sustentar inndefinidamente. O crescimento nesse caso é proporcional a
$p(t)$ e à diferença $K - p(t)$, onde $p(t)$ é o tamanho da população.
Logo $p'(t) = sp(t)(K - p(t)) = sKp(t)(1 - \frac{p(t)}{K})$. Se $sK = r$,
temos o modelo logístico.

\section{Sistema Autônomo}

\section{Modelos das Ciências Naturais}

\end{document}
